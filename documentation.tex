\documentclass[a4paper,12pt]{article}
\usepackage{graphicx}
\graphicspath{ {images/} }
 
%Russian-specific packages
%--------------------------------------
\usepackage[T2A]{fontenc}
\usepackage[utf8]{inputenc}
\usepackage[russian]{babel}
%--------------------------------------
 
%Hyphenation rules
%--------------------------------------
\usepackage{hyphenat}
\hyphenation{ма-те-ма-ти-ка вос-ста-нав-ли-вать}
%--------------------------------------

%Titlepage
%--------------------------------------
\author{Головин Илья}
\title{Разработка справочника для ОС Android}
\date{\today}
%---------------------------------------
 
\begin{document}
\maketitle
\newpage
\tableofcontents
\newpage

\section{Введение}
Товарищи! начало повседневной работы по формированию позиции играет важную роль в формировании новых предложений. С другой стороны новая модель организационной деятельности позволяет выполнять важные задания по разработке системы обучения кадров, соответствует насущным потребностям.
\newpage

\section{Обзор литературы}
При поиске литературы по теме создания довольно конкретного проекта выяснилась одна проблема -- проекты, подобные моему, оказались с закрытым исходным кодом. Понятное дело, что если нет исходного кода, то на литературу можно даже не надеяться. Однако же, это не означает, что литературой я не пользовался вообще -- при создании проекта сильно помогла документация по React Native\footnote{https://reactnative.dev/docs/getting-started}, фреймворку\footnote{Программное обеспечение, облегчающее разработку ПО.}, на основе которого создавался проект. \par
Также, была изучена документация проекта React Native Elements\footnote{https://reactnativeelements.com/docs/}, который предоставляет библиотеку\footnote{Сборник подпрограмм или объектов, используемых для разработки программного обеспечения (ПО).} для типового оформления для программ, написанных на фреймворке React Native. Не менее полезной оказалась документация для библиотеки React Navigation\footnote{https://reactnavigation.org/docs/getting-started}, которая предоставляет способ создавать приложения с множеством различных экранов, которые крайне нужны при создании справочника -- ведь показывать всю информацию на одном экране, как минимум, неудобно для пользователя. \par
В какой-то мере предоставил помощь англоязычный проект tl;drlegal\footnote{https://tldrlegal.com/}, который описывает различные лицензии понятным языком. Благодаря нему, для проекта была выбрана лицензия GNU GPLv3, которая позволяет выпустить проект для публики, не нарушая современное авторское право.
Стоит упомянуть и сам проект GNU\footnote{https://www.gnu.org/gnu/thegnuproject.ru.html} -- там описано, как элегантно указать на то, что приложение выпущено по лицензии GNU GPLv3, а также имелась сокращенная версия этой лицензии, которую можно встроить в приложение, не пугая пользователей огромной стеной текста. 
\end{document}